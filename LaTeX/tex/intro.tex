\chapter{Introduction}
\label{chap:intro}
\paragraph{} Neural network approaches have received a lot of attention in the last several years.
It's becoming a popular choice for performing various tasks like speech and image recognition,
object detections etc. as these methods have dramatically increased accuracy
compared to traditional machine learning approaches.
However, achieving high accuracy on recognition tasks is still computationally
expensive and needs improvements in performance.
This study will be a close resemblance of the recurrent neural network of visual attention which is able to extract necessary
information from an image by looking at it in low resolution, and then
adaptively select parts that are most relevant for a task \cite{DBLP:journals/corr/MnihHGK14}.

 \paragraph{} The idea of visual attention was inspired by how human perception works.
 Humans do not perceive a visual scene as a whole but focus on parts of
 the scene that gives the most useful information to them.
 Humans are also capable of combining information from different parts of a picture.
 They then connect it to build subjective knowledge of the picture (or sequence of pictures) \cite{Goldsborough}.
Taking into account these properties, researchers from google
Deepmind build a model which can be described as follows:
 \blockquote{
 	Instead of processing an entire image or even bounding box at once,
 	at each step, the model selects the next location to attend to;
	based on past information and the demands of the task....
	The model is a recurrent neural network (RNN) which processes
	inputs sequentially, attending to different locations within the
	images (or video frames) one at a time, and incrementally combines
	information from these fixations to build up a dynamic internal
	representation of the scene or environment.\cite{DBLP:journals/corr/MnihHGK14}
}

 \paragraph{} One of the main advantage of this model, is that the computation required is
 controlled and is independent of the input image size.
 Deepmind’s researchers evaluated their model on several image classification
 and dynamic visual control problems which showed a better performance when
 compared with convolution neural network\cite{Goldsborough}.

 \paragraph{} The evidence from this study suggests the application of this model on large scale
 object recognition as well as classification of a sequence of images, which will
 be a great fit since the model's performance is not dependent on the size of an input object.

 \paragraph{} The main aim of this study is to extend the current knowledge of the work mentioned
 above and build a model which will be able to classify a set of images and develop appropriate
 prototype system since it can be useful in a variety of areas. However,
 the experiments in the current work is limited by low-resolution images and mostly will concentrate
 on classifying a group of objects as this restriction will reduce complexity
 of the experiments and therefore reach better results on a task of classifying
 a group of images.

\section{Motivation}
This approach to classify a group of images has a potential to help with automated
detection and classification of breast cancer metastases, which is the main concern
of camelyon challenge. Camelyon challenge provides the whole-slide images (WSI)
of lymph nodes of different patients. Based on this images, competitors solutions
should be able to
detect the presence of breast cancer metastases in lymph nodes. \cite{CAMEL}
\\
Camelyon challenge is an inspiration for this work since pathologist's efforts
along with the assistance of automated detection system will reduce significantly
not only the workload of pathologists but the human error rate in diagnosis as well.

\paragraph{}
This work will be the first step in building software that will be capable of
classifying whole-slide images of histological lymph node at the patient level.
That is, bringing together estimations from multiple lymph node slides into a single outcome.

\paragraph{}
Digital pathology is a very attractive field for machine learning researchers
since whole-slide images have a very high resolution and are typically about 200000 x 100000 pixels.
To give you some sense of data, camelyon challenge provides data for 200 patients,
where each patient has 5 different slides. It means that in total they release
about 1000 slides which is 55.88gb of uncompressed data \cite{CAMEL}.

\paragraph{}
It is quite clear that using \gls{CNN} for this task is computationally very expensive.
Applying model of visual attention promises to solve the issue of
high-resolution pictures at a computational level.
Therefore making an extensible piece of software, that will allow further
improvements is also one of the main concerns of this work.




% 	Hello World!
% 	\subsection{My first sub section}
% 	here goes my first shit
% 	\subsubsection{ My sub sub section}
% 	I will tell you more about neural networks in general and how they will change our life to the better.
% 	\paragraph{ Why NN?}
% 		because it's fucking awesome
% 	\subparagraph{more detailed}
% 	because it's incredibly amazing
% \section{Second section}
% Let's have some math in my document so it will look cooler
% \begin{equation*}
% 	f(*|x) = x^2 / \sqrt{y^{e+1}}
% 	F(x) = y
% \end{equation*}
% \begin{align*}
% 	f(*|x) &= x^2 / \sqrt{y^{e+1}}\\
% 	F(x) = \frac{y^{e+1}}{x^{\frac{x^2}{1}}}
% \end{align*}
% % \left[
% \begin{matrix}
% 1 & 0\\
% 0 & 1
% \end{matrix}
% % \right]
% Some text
% \subsection{My Model}
% \begin{figure}[H]
% 	\includegraphics[width=\linewidth]{trying.png}
% 	\caption{A model.} % to show under the picture
% 	\label{fig:model} % to reference to the figure later
% \end{figure}
% Figure \ref{fig:model} shows a boat.I am so cool that even can type the equation inline $F(x) = P(x|y=1)$
% and still be so awesome
% \begin{table}
% 	\caption{Dummy table}
% \end{table}
% I will cite here something \cite{Ba2015}.
% This is some example text\footnote{\label{myfootnote}Hello footnote}.
% I'm referring to footnote \ref{myfootnote}.
% % Random citation \autocite{Ba2015} embeddeed in text.
% \newpage
%
% \begin{table}[h!]
% 	\centering
% 	\caption{Caption for the table.}
% 	\label{tab:table1}
% 	%  and the letters tell whether we want to align the content to the
% 	% left (l), to the center (c) or to the right (r) for each column.
% 	\begin{tabular}{r||c|r|r}
% 		1 & 2 & 3 & 4\\
% 		\hline
% 		a & b & c & d\\
% 	\end{tabular}
% \end{table}
% I want to refer to the table \ref{tab:table1}
% \newpage
% \bibliographystyle{ieeetr}
% \bibliography{my}
% \printbibliography
% \underline
% \em
%    \emph
