\chapter{Design}
In this chapter we will discuss the architecture of the prototype.
It's divided into two section: Dataset and Model. Since
the dataset is standalone component, this seems to be
a reasonable separation. To make it clear, dataset would be a component
which holds the training, validation and test data including all relevant
functionality that can or should be performed on the
data(this is explained more detailed in \autoref{sec:design_data}).
Model would be a component, which accepts the data(dataset component), performs
training on the training data, and then makes inference on the test dataset.
In other words, model would be everything what is related to building,
training and evaluation of neural networks and reinforcement learning environment.



\section{Dataset}
\label{sec:design_data}
Dataset is a non-real world data, but may behave and represent similar
task as it could in real world data.
\paragraph{explain original MNIST data, what is there, how does this work,
and tf class is used to provide the utility}

\paragraph{Requirements} Like why do we need index generator

\paragraph{Inspiration fromt the tf class dataset}

inpsiration from dataset class from tensor flow: link
% https://github.com/tensorflow/tensorflow/blob/master/tensorflow/contrib/learn/python/learn/datasets/mnist.py
that should perform similar action to
% The function requirement to the dataset can be formulated as

\paragraph{
	not learning any dependency,
	explain that we want to avoid the same samples in a group to avoid that
}
% model will learn dependency and etc.

\paragraph{
	explain shortly about combinations of MNIST data, how to achieve the best data
}

\paragraph{then show the uml diagram of dataset}
including: Class overview

\paragraph{explain what is purpose of each of the class. }

\paragraph{explain what design patter was applied on that with reference to
the Analysis.}

inderect variable access. (e.g.)

\section{Model}
\label{sec:design_model}

\paragraph{Why did you choose picker network?}
\paragraph{about why OO programming is not ALMOST suited and why it should be more functional}
\paragraph{The architecture as a whole}
\paragraph{Possbile classes, overview}
\paragraph{UML diagram}
\paragraph{back up the classes with design patterns}

\paragraph{the flow of the data via function}


\subsection{Configuration of the project, what can be configured, the complexity of the model}





the curent desing is using the approach
described in \autoref{ssec:picker_net}.

Basically explain the flow of input data, all losses, and baselines, and etc.
if there is anything relative to the structure in the code use the references to the design patters
Show all classes, write the documentation for those, or at least briefly explain the purpose.
The process of training, maybe explain the whole idea behind this
	seperation of dataset into validation, test, training.
look at the other bachelorarbeits and look of what they wrote there
think about whether it's possible to combine the design and implemntation chapters



* Entwurfsstrategie, Beschreibungen funktionaler und nich
 funktionaler Anforderungen, Einsatz von Mustern und
 Bibliotheken, Softwarearchitektur, Verwendung von Datentypen und
 Datenstrukturen, Algorithmen, Mensch-Maschine-Schnittstelle;
