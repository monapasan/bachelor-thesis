\chapter{Analysis}

The main objective of this work will be to build an extensible prototype,
which will be built on the model of visual attention but is extended to classify
a set of images. This chapter will explain how we can achieve that.

Firstly, let's define the clear problem statement:
\blockquote{
	Given a dataset where each entry consist of a group of images.
	A certain label is asserted to every entry in a dataset. The goal is
	to build a extensible prototype upon RAM that is capable of classifying this entries.
}

From this statement is clear that prototype should be extensible. Extensible
means that other people (including author itself) are able to extend the
prototype. People should comprehend the code as well as starting and configuring
procedure in order to extend the prototype. Therefore following best practices
to stay consistent with other modern software is an important point of
this work.
\subparagraph{Quality} It is indeed very important to understand what makes

The prototype should be not only a good start for making further improvements
on large scale objects but also be a piece of software that bears the
following properties: extensible, well documented, integrable with other
softwares, simple to configure, easy to start to work with,
readable code.



\subparagraph{Previous work} As we will build the current model upon RAM described
in \autoref{sec:ram_model}, it makes sense to look over the existing implementations.
The authors of RAM paper haven't provide the implementation, but
fortunately there are few implementation built by open source community.
One of them due to certain level of readability is more attractive for this paper.
While the project is way more readable compared to others, it's still in the need
of refactoring.
This code is implemented by using tensorflow framework.

\subparagraph{Tensorflow}

TensorFlow framework has gained a huge popularity among machine learning
community as well as in industry compared to another framework \cite{Goldsborough}.
TensorFlow is an open source software library that makes computations more efcient
by building a computation graph and deploying them to one or more CPUs or GPUs




% in bachelor/implemntation,  http://python-guide-pt-br.readthedocs.io/en/latest/writing/structure/
% there is written a lot about:
% * structure
% * testing
% * pep008

% about why did you choose tensor flow

% In this chapter will be discussed the relation current work
% to previous work

% *Anwendung von Prinzipien, Methoden, Techniken und
% Werkzeugen der Informatik* in einem Anwendungsbereich zum
% Gegenstand haben.

% System- und Anforderungsanalyse, Beschreibungen von
% Systemfunktionen, -dynamik, -daten, -oberfläche,
% Schnittstellendefinitionen, Festlegungen zu Qualitätsparametern

% Bewertung von theoretischen Ansätzen, Konzepten, Methoden,
% Verfahren; informelle
% Aufgabenbeschreibung, klar formulierte Zielstellung;


% introduction to the chapter

% make a clear statement of the problem
% data
% Analysis of the problem
% current data is needed to be extended
% analysis of the existierte projects
% whaat is needed to be done to make the
% what will make the project great
% tested what can be tested, shapewise. Using what?
% tensorboard, the training should be trackable
%


% DATASET
% MNIST dataset is recognised as being the simplest dataset among the neural
% network community, hence building a dataset upon it would be easier. The
% simplicity of dataset would help to understand problems occurring while developing
% a model. There can be different variations for building a group of images to
% classify from MNIST dataset. One of them can be as simple as bringing two
% different numbers together and adding a noise picture, which shouldn't have
% an influence on the outcome. For example: [1,0,2] will be the  first class
% and [0, 3, 1] will be the second class. 0
% represents noise in this example and our model in best case should understand
% that and should not spend much time to explore pictures with noise.



%
% pay attention to Tensorboard since one of the objective of the thesis is to
% make training experience trackable.
% * provide a good overview of outcome. outcome should be as clear as
%  possible.
% * IMPORTANT: you should be able to see in a good and understandable way
%  the selection path of the model. That is, exactly how is it in the original paper.
%  That is, which image is chosen, which part of the image is chosen and etc.
