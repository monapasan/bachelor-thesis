\chapter{Test and evaluation}
In this chapter, we will discuss how the testing is done
in the implementation of this work as well as how
the current model is evaluated.
\section{Test}
In the current implementation, we test our prototype by the means
of unit testing. Unit testing is a method of software testing,
which verifies that individual units of the code are working
as expected. To perform the unit testing, one should identify
pieces of code, that are responsible for a clear task \cite{Huizinga2007}.
In most cases,
that are methods of classes or functions. Imagine, that we have
a function that adds two numbers together. To test this function,
we have to come up with the real values, calculate the expected
return, call the function with the real value and compare
the return of the function with the expected return:

\begin{lstlisting}[language=Python, caption={TensorFlow example \cite{tensorflow2015-whitepaper}},label={list:test_ex}]

	def add(x, y):
		"""Add two numbers together."""
		return x + y

	# unit test of the function `add`
	def test_add():
		assert add(3, 5) == 8 # 3, 5 are real values, 8 is the expected return
		assert add(10, 2) == 12
\end{lstlisting}

For testing purposes, we
used unit test framework called Nose. Nose simplifies unit testing
by providing automatic test discovery. It also automatically creates a test suite,
which is nothing more than a set of tests. One advantage of using the Nose
over the Python’s standard unit test module is that no configuration is required
in order to run all tests when the tests are named consistent with Nose conventions
and are located in the \emph{\\tests} directory. An additional advantage is that
Nose is generating well-readable report after running tests, that helps
when debugging and fixing tests.
\subparagraph{} The module Dataset is fully tested with unit tests. However,
the module Model is not, since it has too many non-deterministic values, which
is way harder to test.

\section{Evaluation} The model was evaluated with the dataset described
in \autoref{sec:analysis_dataset} on server from HTW Berlin.

Results of evaluation
